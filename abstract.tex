% +--------------------------------------------------------------------+
% | Copyright Page
% +--------------------------------------------------------------------+

\cleardoublepage

\thispagestyle{empty}

\begin{center}

{\bf \Huge Abstract}

  \end{center}
\vspace{1cm}

Asymmetric architectures, composed by multiple processors with the same
instruction set but different performance and energy consumption
characteristics, provide many optimisation possibilities in performance
and/or energy consumption over the execution of parallel
applications. Scheduling tasks in this kind of architectures in such a way
that all the resources are used efficiently is a difficult task, and it is
usually addressed using parallel programming models, allowing programmers
to annotate the parallelism for each of the different tasks of
the program, and runtime environments which dinamically take advantage of
this kind of parallelism.

% Las arquitecturas asimétricas, formadas por varios procesadores con el
% mismo repertorio de instrucciones pero distintas características de
% rendimiento y consumo, ofrecen muchas posibilidades de optimización del
% rendimiento y/o el consumo en la ejecución de aplicaciones paralelas. La
% planificación de tareas sobre dichas arquitecturas de forma que se
% aprovechen de manera eficiente los distintos recursos, es muy compleja y se
% suele abordar utilizando modelos de programación paralelos, que permiten al
% programador especificar el paralelismo de las tareas, y entornos de
% ejecución que explotan dinámicamente dicho paralelismo.


In this work we have modified one of the most used task schedulers nowadays
to try to use all the resources efficiently if high performance is needed,
or to achieve the best energy efficiency if consumption reduction is the
priority. A math library developed specifically for this kind of asymmetric
architectures in the University of Texas, Austin, has been used as well.

% En este trabajo hemos modificado uno de los planificadores de tareas más
% utilizados en la actualidad para intentar aprovechar todos los recursos al
% máximo, cuando el rendimiento así lo necesite, o para conseguir la mejor
% eficiencia energética posible, cuando el consumo sea más
% prioritario. También se ha utilizado una biblioteca desarrollada
% específicamente para la arquitectura asimétrica objeto del estudio en la
% Universidad de Austin.


To obtain maximum performance, cores have been grouped into two different
levels: a symmetric cluster with identical virtual cores where each of them
is composed by a set of asymmetric cores. With this approach, schedulers
can allocate tasks to the virtual cores in the same way as they would do in
a symmetric multicore system, and the library is responsible for the
allocation of tasks to the asymmetric cores. The work carried out consisted
of integrating this library with the task scheduler.

% Para obtener el máximo rendimiento se han agrupado los núcleos del sistema
% en dos niveles: hay un cluster simétrico de núcleos virtuales idénticos,
% cada uno de los cuales está compuesto por un conjunto de núcleos
% asimétricos. El planificador de tareas asigna trabajo a los núcleos
% virtuales, de manera idéntica a como lo haría en un sistema multinúcleo
% simétrico, y la biblioteca se encarga de repartir el trabajo entre los
% núcleos asimétricos. El trabajo ha consistido en integrar dicha biblioteca
% con el planificador de tareas.

To improve energy efficiency, new policies have been included into the
scheduler. These policies have been designed to take advantage of the low
consumption modes of the architecture and to disable or not assign tasks to
some of the cores, activating them at run-time if the execution does not
need all the available resources in the architecture.

% Para mejorar la eficiencia energética se han incluido en el
% planificador de tareas políticas de explotación de los modos de bajo
% consumo de la arquitectura y también de apagado o no asignación de carga de
% trabajo a algunos de los núcleos, que se activan en tiempo de ejecución
% cuando se detecta que la aplicación no necesita todos los recursos
% disponibles en la arquitectura.

\begin{center}
  {\bf \Large Keywords}
  
\end{center}
{
\parindent=0in
Asymmetric Architectures, scheduling, DVFS, performance improvement, energy
consumption.
}


%-- Configuraciones para emacs --
%%% Local Variables:
%%% mode: latex
%%% TeX-master: "./principal.tex"
%%% End:

   



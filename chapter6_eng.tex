\cleardoublepage

\chapter{Conclusions}
\label{ch:chapter6}

\section{Conclusions and future work}


Throughout this work, we have presented a set of newly developed techniques
over asymmetric architectures, focusing on ARM designed big.LITTLE
systems-on-chip, which are aimed at obtaining performance and energy
efficiency improvements in parallel applications.

% En este trabajo se han desarrollado un conjunto de técnicas sobre
% arquitecturas asimétricas, y en especial para los procesadores big.LITTLE
% diseñados por ARM, que buscan obtener mejoras tanto en el rendimiento de las
% aplicaciones como en la eficiencia energética de la plataforma.


We have demonstrated that, for a task-level parallelism model, an approach
that delegates the burden of dealing with asymmetry to the library (in our
case, using an asymmetry-aware BLIS implementation), does not require any
formulation of an existing task scheduler, and can deliver high
performance. This proposal paves the road towards reusing conventional
runtime schedulers for SMPs (and all the associated improvement techniques
developed through the past few years), as the runtime only has a symmetric
view of the hardware, at the cost of developing asymmetry-aware underlying
libraries. Our experiments reveal that this solution is competitive and
even improves the results obtained with an asymmetry-aware scheduler for
DLA operations. This work has been presented in \emph{``The Sixth
  International Workshop on Accelerators and Hybrid Exascale Systems
  (AsHES)''}, with the paper~\cite{ashes} shown in the bibliography.


% En base a un paradigma basado en la extracción de paralelismo a nivel de
% tareas en tiempo de ejecución, y a través de una versión de la biblioteca
% BLIS adaptada a arquitecturas asimétricas, y basándose en un planificador
% de tareas convencional, se ha demostrado como es posible obtener un mejor
% rendimiento sin necesidad de adaptar el código del problema a la
% arquitectura, o introducir cambios específicos en el planificador.  Esta
% propuesta permite reutilizar los \emph{runtimes} convencionales (y todas
% las técnicas desarrolladas durante los últimos años) sobre las nuevas
% arquitecturas asimétricas con el único coste de la implementación de
% versiones asimétricas de cada rutina de la librería.  Los experimentos
% realizados han demostrado que esta solución es totalmente comparable con un
% planificador consciente de la asimetría de la plataforma, incluso llegando
% a obtener mejores resultados en el rendimiento de la aplicación. Esta
% técnica ha sido presentada en \emph{``The Sixth International Workshop on
%   Accelerators and Hybrid Exascale Systems (AsHES)''}, a través del
% artículo~\cite{ashes} mostrado en la bibliografía.



Trying to improve energy efficiency, a set of new policies has been
integrated into the asymmetry-aware scheduler, based both on dynamic
voltage frequency scaling (DVFS) techniques and scheduling techniques,
making decisions dynamically depending on the number of ready tasks to be
executed every time. The first set of experiments shows how, independently
of the platform used, applying techniques of frequency scaling over \BIG
cores leads to better results than an asymmetry-aware scheduler.

% Buscando la mejora de la eficiencia energética, se han desarrollado una
% serie de políticas integradas sobre un planificador consciente de la
% asimetría, basadas tanto en técnicas de escalado de frecuencia (DVFS) como
% en técnicas de planificación, tomando las decisiones de forma dinámica en
% función del número de tareas listas para ser ejecutadas en cada momento. El
% primer conjunto de experimentos realizados muestran como,
% independientemente de las dos plataformas sobre las que se han ejecutado
% los experimentos, aplicar técnicas de escalado de frecuencia sobre el
% cluster de núcleos \BIG obtiene una mejora en el rendimiento energético
% significativa frente a un planificador convencional consciente de la
% asimetría sin ninguna política de ahorro energético integrada.


Similarly, the second set of experiments shows how, in processors which
allow deactivating a full cluster at \emph{hardware} level, switching
it off at certain times during the execution when the number of ready tasks
is low enough allows to obtain an energy performance gain by drastically
reducing the average instant power consumed. Also, these experiments reveal
that, in architectures that do not support a \emph{hardware} switching off,
a policy which does not assign tasks to a certain cluster has the same
energy performance results that an asymmetric-aware scheduler has,
decreasing the average power consumed.

% De igual manera, el segundo conjunto de experimentos revela como en
% aquellos procesadores que permiten desactivar un cluster completo a nivel
% \emph{hardware}, apagarlo en ciertos momentos de la ejecución cuando el
% nivel de tareas listas es lo suficientemente bajo permite obtener una
% mejora de rendimiento energético reduciendo la potencia instantánea media
% consumida drásticamente. También se ha puesto de manifiesto como, en los
% procesadores que no soportan un apagado del cluster a nivel físico, una
% política de planificación de tareas que no asigne tareas a ese cluster
% obtiene los mismos resultados de eficiencia energética que el planificador
% normal consciente de la asimetría.






As future work, we are considering to carry out the experiments over a bigger
set of problems of scientific interest, and to run them over other
platforms where the number of \BIG and \LITTLE cores is less
balanced. Additionally, we consider to develop new techniques that make
decisions dynamically depending on the problem we are trying to solve and
its associated DAG, as well as partial results taken during previous task
executions.


% Como trabajo futuro, se considera realizar los experimentos sobre un
% conjunto mayor de problemas de interés científico, así como sobre otras
% arquitecturas donde el número de núcleos big y LITTLE se encuentre más
% desequilibrado. Adicionalmente, se considera desarrollar políticas que
% tomen decisiones de manera dinámica en base al problema ejecutado y su DAG
% correspondiente, así como a resultados parciales tomados durante la
% ejecución de las distintas tareas previas del problema.



%-- Configuraciones para emacs --
%%% Local Variables:
%%% mode: latex
%%% TeX-master: "./principal.tex"
%%% End:

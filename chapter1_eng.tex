\cleardoublepage
% \newpage
% \thispagestyle{empty}
% \mbox{}


\chapter{Introduction}
\label{ch:chapter1}

\section{Motivation}

En los últimos años, el consumo energético ha emergido como una de las
principales barreras que ha frenado la mejora de rendimiento en los
sistemas de cómputo, desde dispositivos móviles hasta grandes centros de
procesamiento de datos (CPDs). Con el fin de mejorar la eficiencia
energética, resulta necesario explotar mayores grados de especialización en
el hardware. Las arquitecturas actuales proporcionan diversos tipos de
heterogeneidad, que van desde configuraciones heterogéneas en CPDs, hasta
procesadores con aceleradores hardware integrados en el propio chip,
plataformas híbridas formadas por procesadores de propósito general (CPUs)
y aceleradores de propósito específico (como pueden ser GPUs, Intel Xeon
Phi, etc.), multicores asimétricos (AMPSs) con repertorio común de
instrucciones o multicores heterogéneos con múltiples repertorios de
instrucciones~\cite{NRC2011,KoSh13}.



La correcta explotación del rendimiento y la eficiencia energética
ofrecidos por estas arquitecturas conlleva, de forma necesaria, un
incremento en la complejidad del software que permita, de forma lo más
transparente posible para el usuario final, la adaptación de aplicaciones
ya existentes o de nuevo desarrollo para aumentar el rendimiento y reducir
su consumo energético~\cite{OsTo10,SKC+15}. Uno de los principales retos en
este sentido es la correcta asignación de trabajo a los recursos
computacionales existentes, de modo que, de forma simultánea, se optimice
su grado de ocupación y la adaptación del recurso escogido a la tarea
específica a desarrollar. En respuesta a estas necesidades, una de las
soluciones propuestas por la comunidad científica es el desarrollo de
modelos de programación que expongan el paralelismo a nivel de tareas,
explotado en tiempo de ejecución por planificadores de tareas (comúnmente
conocidos como runtimes~\cite{VMC+14}). Esta capa de software es capaz de
gestionar de forma transparente al usuario los procesos de gestión de
dependencias de datos entre tareas, las transferencias de datos en entornos
heterogéneos, y el mapeado eficiente entre tipos de tareas y de procesador,
entre otros aspectos.

Atendiendo a las arquitecturas subyacentes, el trabajo se centra en una de
las familias de arquitecturas con más perspectiva de futuro de entre las
anteriormente mencionadas: procesadores multinúcleo asimétricos (AMPs) de
bajo consumo, y en concreto, procesadores de la familia big.LITTLE de ARM.
Este tipo de configuraciones ha suscitado gran interés en el mercado de las
aplicaciones y ecosistemas móviles, debido a la especialización de cada
tipo de procesador ante tareas concretas. Esta especialización se suele
traducir en la activación/desactivación de cada tipo de núcleo en función
de la demanda computacional o prioridad de las aplicaciones en ejecución, o
de las limitaciones energéticas ligadas a cada escenario concreto de
ejecución.


\section{Objectives}
Desde el punto de vista de la computación de altas prestaciones (HPC), el
uso conjunto de todos los recursos computacionales desde una misma
aplicación multihebra resulta de especial interés. Aunque algunas de las
técnicas ya desarrolladas en bibliotecas o planificadores de tareas
existentes pueden ser aplicadas directamente en este tipo de arquitecturas,
éstas exponen nuevos retos y problemáticas que hay que abordar. Este
trabajo se centra en dos de estos retos:
\begin{itemize}
\item Integración y combinación de bibliotecas específicas conscientes de
  la asimetría con \emph{runtimes} convencionales (no conscientes de la
  asimetría), y comparativa de este enfoque con un diseño de \emph{runtime}
  sí consciente con la asimetría de la arquitectura.

\item Diseño de técnicas de reducción de consumo energético. Aplicación de
  técnicas de escalado dinámico de frecuencia (DVFS) o desactivación de
  núcleos en distintas fases de la ejecución de una determinada aplicación,
  en función de las demandas de cómputo y grado de paralelismo disponible, y
  evaluación del impacto de este tipo de técnicas en el tiempo de ejecución
  y consumo energético global.
\end{itemize}


\section{Methodology and work plan}
Para alcanzar los objetivos descritos en el apartado anterior, se han
abordado las siguientes tareas específicas:
\begin{enumerate}[T1.]
\item Estudio del estado del arte en sistemas asimétricos desde el punto de
  vista arquitectural, y del soporte software disponible para explotar sus
  características, a nivel de sistema operativo, \emph{runtime} o
  aplicación.
\item Estudio, análisis y evaluación del rendimiento de familias de
  bibliotecas y planificadores de tareas en tiempo de ejecución no
  conscientes de la asimetría de la arquitectura, así como soluciones ya
  adaptadas a esta familia de arquitecturas. En el ámbito del álgebra
  lineal, utilizado como hilo conductor del trabajo, se ha utilizado la
  biblioteca BLIS~\cite{BLIS1}.
\item Integración de las anteriores bibliotecas sobre \emph{runtimes} no
  conscientes de la asimetría ya existentes, para explotar de forma
  transparente la asimetría de las arquitecturas sin modificaciones en los
  planificadores, para un conjunto de operaciones matemáticas de interés.
\item Diseño de heurísticas, políticas de asignación de recursos y
  explotación de los modos de bajo consumo proporcionados por las
  arquitecturas, que permitan un aprovechamiento óptimo de los recursos a
  nivel del planificador de tareas en \emph{runtimes} existentes, con el
  objetivo de explotar de manera energéticamente eficiente todos los
  recursos computacionales existentes.
\item Evaluación del rendimiento y eficiencia energética de las nuevas
  políticas desarrolladas. Comparación de los resultados con soluciones
  actuales, tanto conscientes como no conscientes de la asimetría de la
  arquitectura.
\end{enumerate}

\section{Document structure}
La presente memoria se estructura como sigue. 
El Capítulo~\ref{ch:chapter2} ofrece una descripción general de las
arquitecturas heterogéneas y asimétricas actualmente más utilizadas en el
ámbito de la computación de altas prestaciones, haciendo especial hincapié
en el paradigma big.LITTLE de ARM utilizado en el trabajo
realizado. Además, detalla las características específicas del entorno
experimental empleado a nivel hardware y software.
El Capítulo~\ref{ch:chapter3} introduce dos de las estrategias de
paralelización utilizadas en el desarrollo del trabajo: extracción de
paralelismo a nivel de tareas, y extracción de paralelismo a nivel de
datos, centrándose en las características de cada uno, así como en sus
ventajas e inconvenientes.
Los Capítulos~\ref{ch:chapter4} y~\ref{ch:chapter5} describen en detalle el
trabajo realizado, dividido en la propuesta y evaluación de técnicas para
la mejora de rendimiento en planificadores a nivel de tareas, y en el
desarrollo de soluciones para la mejora de la eficiencia energética,
respectivamente.
Finalmente, el Capítulo~\ref{ch:chapter6} incluye una serie de conclusiones
generales y líneas de trabajo futuras.


%-- Configuraciones para emacs --
%%% Local Variables:
%%% mode: latex
%%% TeX-master: "./principal.tex"
%%% End:

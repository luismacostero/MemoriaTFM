\cleardoublepage
\chapter{Conclusiones}
\label{ch:chapter6}

\section{Conclusiones y trabajo futuro}

En este trabajo se han desarrollado un conjunto de técnicas sobre
arquitecturas asimétricas, y en especial para los procesadores big.LITTLE
diseñados por ARM, que buscan obtener mejoras tanto en el rendimiento de las
aplicaciones como en la eficiencia energética de la plataforma.

En base a un paradigma basado en la extracción de paralelismo a nivel de
tareas en tiempo de ejecución, y a través de una versión de la biblioteca
BLIS adaptada a arquitecturas asimétricas, y basándose en un planificador
de tareas convencional, se ha demostrado como es posible obtener un mejor
rendimiento sin necesidad de adaptar el código del problema a la
arquitectura, o introducir cambios específicos en el planificador. Esta
propuesta permite reutilizar los \emph{runtimes} convencionales (y todas
las técnicas desarrolladas durante los últimos años) sobre las nuevas
arquitecturas asimétricas con el único coste de la implementación de
versiones asimétricas de cada rutina de la librería.  Los experimentos
realizados han demostrado que esta solución es totalmente comparable con un
planificador consciente de la asimetría de la plataforma, incluso llegando
a obtener mejores resultados en el rendimiento de la aplicación. Esta
técnica ha sido presentada en \emph{``The Sixth International Workshop on
  Accelerators and Hybrid Exascale Systems (AsHES)''}, a través del
artículo~\cite{ashes} mostrado en la bibliografía.

Buscando la mejora de la eficiencia energética, se han desarrollado una
serie de políticas integradas sobre un planificador consciente de la
asimetría, basadas tanto en técnicas de escalado de frecuencia (DVFS) como
en técnicas de planificación, tomando las decisiones de forma dinámica en
función del número de tareas listas para ser ejecutadas en cada momento. El
primer conjunto de experimentos realizados muestra como,
independientemente de las dos plataformas sobre las que se han ejecutado
los experimentos, aplicar técnicas de escalado de frecuencia sobre el
cluster de núcleos \BIG obtiene una mejora en el rendimiento energético
significativa frente a un planificador convencional consciente de la
asimetría sin ninguna política de ahorro energético integrada.

De igual manera, el segundo conjunto de experimentos revela como en
aquellos procesadores que permiten desactivar un cluster completo a nivel
\emph{hardware}, apagarlo en ciertos momentos de la ejecución cuando el
nivel de tareas listas es lo suficientemente bajo permite obtener una
mejora de rendimiento energético reduciendo la potencia instantánea media
consumida drásticamente. También se ha puesto de manifiesto como, en los
procesadores que no soportan un apagado del cluster a nivel físico, una
política de planificación de tareas que no asigne tareas a ese cluster
obtiene los mismos resultados de eficiencia energética que el planificador
normal consciente de la asimetría.


Como trabajo futuro, se considera realizar los experimentos sobre un
conjunto mayor de problemas de interés científico, así como sobre otras
arquitecturas donde el número de núcleos big y LITTLE se encuentre más
desequilibrado. Adicionalmente, se considera desarrollar políticas que
tomen decisiones de manera dinámica en base al problema ejecutado y su DAG
correspondiente, así como a resultados parciales tomados durante la
ejecución de las distintas tareas previas del problema.



%-- Configuraciones para emacs --
%%% Local Variables:
%%% mode: latex
%%% TeX-master: "./principal.tex"
%%% End:

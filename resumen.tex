% +--------------------------------------------------------------------+
% | Copyright Page
% +--------------------------------------------------------------------+

\cleardoublepage
\thispagestyle{empty}

\begin{center}
  {\bf \Huge Resumen}
\end{center}
%\vspace{0.5cm}

Las arquitecturas asimétricas, formadas por varios procesadores con el
mismo repertorio de instrucciones pero distintas características de
rendimiento y consumo, ofrecen muchas posibilidades de optimización del
rendimiento y/o el consumo en la ejecución de aplicaciones paralelas. La
planificación de tareas sobre dichas arquitecturas de forma que se
aprovechen de manera eficiente los distintos recursos, es muy compleja y se
suele abordar utilizando modelos de programación paralelos, que permiten al
programador especificar el paralelismo de las tareas, y entornos de
ejecución que explotan dinámicamente dicho paralelismo.

En este trabajo hemos modificado uno de los planificadores de tareas más
utilizados en la actualidad para intentar aprovechar todos los recursos al
máximo, cuando el rendimiento así lo necesite, o para conseguir la mejor
eficiencia energética posible, cuando el consumo sea más
prioritario. También se ha utilizado una biblioteca desarrollada
específicamente para la arquitectura asimétrica objeto de estudio en la
Universidad de Texas, Austin.

Para obtener el máximo rendimiento se han agrupado los núcleos del sistema
en dos niveles: hay un cluster simétrico de núcleos virtuales idénticos,
cada uno de los cuales está compuesto por un conjunto de núcleos
asimétricos. El planificador de tareas asigna trabajo a los núcleos
virtuales, de manera idéntica a como lo haría en un sistema multinúcleo
simétrico, y la biblioteca se encarga de repartir el trabajo entre los
núcleos asimétricos. El trabajo ha consistido en integrar dicha biblioteca
con el planificador de tareas.

Para mejorar la eficiencia energética se han incluido en el planificador de
tareas políticas de explotación de los modos de bajo consumo de la
arquitectura y también de apagado o no asignación de carga de trabajo a
algunos de los núcleos, que se activan en tiempo de ejecución cuando se
detecta que la aplicación no necesita todos los recursos disponibles en la
arquitectura.


\begin{center}
  {\bf \Large Palabras clave}
\end{center}
{
\parindent=0in   
Arquitecturas asimétricas, planificación, DVFS, mejora rendimiento, consumo
energético.
}



%-- Configuraciones para emacs --
%%% Local Variables:
%%% mode: latex
%%% TeX-master: "./principal.tex"
%%% End:

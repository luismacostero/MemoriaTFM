\cleardoublepage
% \newpage
% \thispagestyle{empty}
% \mbox{}


\chapter{Introducción}
\label{ch:chapter1}

\section{Motivación}
En los últimos años, el consumo energético ha emergido como una de las
principales barreras que ha frenado la mejora de rendimiento en los
sistemas de cómputo, desde dispositivos móviles hasta grandes centros de
procesamiento de datos. Con el fin de mejorar la eficiencia energética,
resulta necesario explotar mayores grados de especialización en el
hardware. Las arquitecturas actuales proporcionan diversos tipos de
heterogeneidad, que van desde configuraciones heterogéneas en CPDs, hasta
procesadores con aceleradores hardware integrados en el propio chip,
plataformas híbridas formadas por procesadores de propósito general (CPUs)
y aceleradores de propósito específico (como pueden ser GPUs, Intel Xeon
Phi, etx.), multicores asimétricos (AMPSs) con repertorio común de
instrucciones o multicores heterogéneos con múltiples repertorios de
instrucciones~\cite{FuMi11,KoSh13}.

La correcta explotación del rendimiento y la eficiencia energética
ofrecidos por estas arquitecturas conlleva, de forma necesaria, un
incremento en la complejidad del software que permita, de forma lo más
transparente posible para el usuario final, la adaptación de aplicaciones
ya existentes o de nuevo desarrollo para aumentar el rendimiento y reducir
su consumo energético~\cite{OsTo10,SKC+15}. Uno de los principales retos en
este sentido es la correcta asignación de trabajo a los recursos
computacionales existentes, de modo que, de forma simultánea, se optimice
su grado de ocupación y la adaptación del recurso escogido a la tarea
específica a desarrollar. En respuesta a estas necesidades, una de las
soluciones propuestas por la comunidad científica es el desarrollo de
modelos de programación que expongan el paralelismo a nivel de tareas,
explotado en tiempo de ejecución por planificadores de tareas (comúnmente
conocidos como runtimes [VMC+14]). Esta capa de software es capaz de
gestionar de forma transparente al usuario los procesos de gestión de
dependencias de datos entre tareas, las transferencias de datos en entornos
heterogéneos, y el mapeado eficiente entre tipos de tareas y de procesador,
entre otros aspectos.

Atendiendo a las arquitecturas subyacentes, el trabajo se centra en una de
las familias de arquitecturas con más perspectiva de futuro de entre las
anteriormente mencionadas: procesadores multinúcleo asimétricos (AMPs) de
bajo consumo, y en concreto, procesadores de la familia big.LITTLE de ARM.
Este tipo de configuraciones ha suscitado gran interés en el mercado de las
aplicaciones y ecosistemas móviles, debido a la especialización de cada
tipo de procesador ante tareas concretas. Esta especialización se suele
traducir en la activación/desactivación de cada tipo de núcleo en función
de la demanda computacional o prioridad de las aplicaciones en ejecución, o
de las limitaciones energéticas ligadas a cada escenario concreto de
ejecución.



\section{Objetivos}



\section{Metodología y plan de trabajo}

\section{Estructura del documento}

La presente memoria se estructura como sigue. 
El Capítulo~\ref{ch:chapter2} ofrece una descripción general de las arquitecturas heterogéneas
y asimétricas actualmente más utilizadas en el ámbito de la computación de altas prestaciones,
haciendo especial hincapié en el paradigma big.LITTLE de ARM utilizado en el trabajo realizado. Además,
detalla las características específicas del entorno experimental empleado a nivel hardware y software.
El Capítulo~\ref{ch:chapter3} introduce dos de las estrategias de paralelización utilizadas en el
desarrollo del trabajo: extracción de paralelismo a nivel de tareas, y extracción de paralelismo
a nivel de datos, centrándose en las características de cada uno, así como en sus ventajas e inconvenientes.
Los Capítulos~\ref{ch:chapter4} y~\ref{ch:chapter5} describen en detalle el trabajo realizado, dividido
en la propuesta y evaluación de técnicas para la mejora de rendimiento en planificadores a nivel de tareas,
y en el desarrollo de soluciones para la mejora de la eficiencia energética, respectivamente.
Finalmente, el Capítulo~\ref{ch:chapter6} incluye una serie de conclusiones generales y líneas
de trabajo futuras.


%-- Configuraciones para emacs --
%%% Local Variables:
%%% mode: latex
%%% TeX-master: "./principal.tex"
%%% End:

\cleardoublepage
% \newpage
% \thispagestyle{empty}
% \mbox{}


\chapter{Introducción}
\label{ch:chapter1}

\section{Motivación}

\section{Objetivos}

\section{Metodología y plan de trabajo}

\section{Estructura del documento}

La presente memoria se estructura como sigue. 
El Capítulo~\ref{ch:chapter2} ofrece una descripción general de las arquitecturas heterogéneas
y asimétricas actualmente más utilizadas en el ámbito de la computación de altas prestaciones,
haciendo especial hincapié en el paradigma big.LITTLE de ARM utilizado en el trabajo realizado. Además,
detalla las características específicas del entorno experimental empleado a nivel hardware y software.
El Capítulo~\ref{ch:chapter3} introduce dos de las estrategias de paralelización utilizadas en el
desarrollo del trabajo: extracción de paralelismo a nivel de tareas, y extracción de paralelismo
a nivel de datos, centrándose en las características de cada uno, así como en sus ventajas e inconvenientes.
Los Capítulos~\ref{ch:chapter4} y~\ref{ch:chapter5} describen en detalle el trabajo realizado, dividido
en la propuesta y evaluación de técnicas para la mejora de rendimiento en planificadores a nivel de tareas,
y en el desarrollo de soluciones para la mejora de la eficiencia energética, respectivamente.
Finalmente, el Capítulo~\ref{ch:chapter6} incluye una serie de conclusiones generales y líneas
de trabajo futuras.


%-- Configuraciones para emacs --
%%% Local Variables:
%%% mode: latex
%%% TeX-master: "./principal.tex"
%%% End:
